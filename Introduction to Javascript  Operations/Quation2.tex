
In JavaScript, operators are categorized based on their functionality. These categories include

Arithmetic Operators:

Arithmetic operators are used to perform mathematical calculations.
Examples include addition (+), subtraction (-), multiplication (*), division (/), and modulus (%).


let a = 5;
let b = 2;
let sum = a + b; // Addition
let difference = a - b; // Subtraction
let product = a * b; // Multiplication
let quotient = a / b; // Division
let remainder = a % b; // Modulus


Assignment Operators:

Assignment operators are used to assign values to variables.
Examples include equals (=), addition assignment (+=), subtraction assignment (-=), multiplication assignment (*=), and division assignment (/=).

let x = 10;
x += 5; // Equivalent to x = x + 5;


Comparison Operators:

Comparison operators are used to compare values and return a Boolean result (true or false).
Examples include equal to (==), not equal to (!=), greater than (>), less than (<), greater than or equal to (>=), and less than or equal to (<=).

let num1 = 10;
let num2 = 5;
let isEqual = num1 == num2; // false
let isGreater = num1 > num2; // true


Logical Operators:

Logical operators are used to combine conditional statements.
Examples include logical AND (&&), logical OR (||), and logical NOT (!)let isSunny = true;
let isWarm = true;
let isGoodWeather = isSunny && isWarm; // true


Bitwise Operators:

Bitwise operators are used to perform bitwise operations on integer operands.
Examples include bitwise AND (&), bitwise OR (|), bitwise XOR (^), left shift (<<), and right shift (>>).

var num1 = 5; // 0101
var num2 = 3; // 0011
var result = num1 & num2; // Bitwise AND: 0001 (1 in decimal)


Unary Operators:

Unary operators act on a single operand.
Examples include unary plus (+), unary minus (-), increment (++), and decrement (--).

var num = 5;
num++; // Increment